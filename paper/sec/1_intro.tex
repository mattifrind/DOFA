\section{Introduction}
\label{sec:intro}

The success of computer vision deep learning models relies heavily on the large amount of data available for training on the internet \cite{8237359}.
In the domain of remote sensing though, labeled data is hard to obtain because labeling is a complicated task (requires expert knowledge) and there is a much larger range of sensors \cite{dofa}.
There are multiple ways to address this issue like synthetic data or semi-supervised learning algorithms. After the breakthroughs of LLMs and the concept of pretraining foundation models on a wide range of tasks, recent research tried to transfer this concept in the area of remote sensing computer vision \cite{bommasani2022opportunitiesrisksfoundationmodels}.

This report focuses on one foundational model called "DOFA" \cite{dofa}
and evaluates it on a new downstream task for comparison with other foundation models. The used dataset is called BigEarthNet \cite{bigearthnet}
and we will be using the Sentinel-1 data. We will provide results with different metrics on the multi-label classification for the 19-class variant and also the 43-class variant.

We will analyze the meaningfulness of the features computed by the DOFA model by applying a UMAP transformation \cite{umap-paper}
to understand the meaningfulness of the features. Afterward, we use the features as input data for training different classifiers on the classification task.

% TODO: example images